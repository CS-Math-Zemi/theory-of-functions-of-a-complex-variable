\documentclass[../main.tex]{subfiles}

% =============================================
% 【自動設定】単体ビルド時の章番号補正
% ファイル名 (chXX) の数字を読み取り、自動でセクション番号を調整します
% main.tex からビルドする時は無視されます
% =============================================
\directlua{
  local s, e, num = string.find(tex.jobname, "ch(" .. string.char(37) .. "d+)")%chktex 36 %chktex 18 %chktex 12 %chktex 26 %chktex 8
  if num then
    tex.print(string.char(92) .. "setcounter{section}{" .. (tonumber(num) - 1) .. "}")%chktex 36 %chktex 18 %chktex 12 %chktex 26 %chktex 8
  end
}

\begin{document}
\begin{flushleft}

\section{複素関数の微分}

\subsection{滑らかさ}
複素関数 $f$ を
\begin{mathpad}
  \[
    f(z)\:=\:u(x,y)+iv(x,y)\quad (z=x+iy)
  \]
\end{mathpad}
と書き, 実部 $u$, 虚部 $v$ を実二変数関数として扱う. 以降は $u$ と $v$ の偏導関数を用いて, 複素関数の滑らかさを整理する. 

\begin{definition}{偏導関数}{def:partial-derivative}
  点 $(x,y)$ において
  \begin{mathpad}
    \[
      \pdv{u}{x}(x,y)\:=\:\lim_{\Delta x\to 0}\dfrac{u(x+\Delta x,y)-u(x,y)}{\Delta x},\quad
      \pdv{u}{y}(x,y)\:=\:\lim_{\Delta y\to 0}\dfrac{u(x,y+\Delta y)-u(x,y)}{\Delta y}
    \]
  \end{mathpad}
  が存在するとき, それぞれを $u$ の $x$・$y$ 方向の偏導関数という. $v$ についても同様に定義する. 
\end{definition}

ここでのポイントは, $x$ 方向・$y$ 方向の「切り分け」である. つまり $y$ を固定して $x$ だけを動かすときの変化が $u_x$, $x$ を固定して $y$ だけを動かすときの変化が $u_y$ である. 
微小な変化に対しては
\begin{mathpad}
  \[
    u(x+\Delta x,y+\Delta y)\approx u(x,y)+u_x\Delta x+u_y\Delta y
  \]
\end{mathpad}
と近似できる. ここで $u_x$ は $y$ を固定した断面における $x$ 方向の傾きであり, $u_x\Delta x$ は $x$ 方向に $\Delta x$ だけ動いたときの変化量を表している. 
同様に $u_y$ は $x$ を固定した断面の $y$ 方向の傾きで, $u_y\Delta y$ は $y$ 方向の変化量を表す. 
したがって $u_x\Delta x+u_y\Delta y$ は, $x$ 方向と $y$ 方向の変化量を足し合わせた一次近似になっていると理解できる. 
同様に $v$ の偏導関数は虚部の変化量を同じ形で与える. 

\begin{figure}[H]
  \centering
\includegraphics[width=0.9\linewidth]{img/fig4-0.pdf}%chktex 8
  \caption{偏導関数と一次近似の幾何的意味}\label{fig:fig4-0}
\end{figure}

\begin{definition}{$C^1$ 級, $C^n$ 級}{def:cn}
  領域 $D\subset\mathbb{C}$ で $u,\:v$ の偏導関数 $u_x,\:u_y,\:v_x,\:v_y$ が存在して連続であるとき, $f$ は $D$ 上で $C^1$ 級であるという. 
  さらに任意の $k\leq n$ について $k$ 階の偏導関数が存在して連続であるとき, $f$ は $D$ 上で $C^n$ 級であるという. 
\end{definition}

\begin{example}{2乗関数の滑らかさ}{ex:smooth-square}
  $f(z)\:=\:z^2$ とすると
  \begin{mathpad}
    \[
      u(x,y)\:=\:x^2-y^2,\quad v(x,y)\:=\:2xy
    \]
  \end{mathpad}
  である. よって
  \begin{mathpad}
    \[
      u_x=2x,\quad u_y=-2y,\quad v_x=2y,\quad v_y=2x
    \]
  \end{mathpad}
  は全て連続であり, $f$ は $\mathbb{C}$ 上で $C^1$ 級である. 
\end{example}

\subsection{全微分可能性}
複素関数を $f(x,y)\:=\:(u(x,y),v(x,y))$ とみなし, $\mathbb{R}^2\to\mathbb{R}^2$ の写像として微分可能性を考える. 

\begin{definition}{全微分可能性}{def:total-differentiability}
  点 $(x,y)$ において, ある線形写像 $L\colon\mathbb{R}^2\to\mathbb{R}^2$ が存在して
  \begin{mathpad}
    \[
      f(x+\Delta x,y+\Delta y)-f(x,y)\:=\:L(\Delta x,\Delta y)+o(r)
    \]
  \end{mathpad}
  が $r\:=\:\sqrt{(\Delta x){}^2+(\Delta y){}^2}\to 0$ で成り立つとき, $f$ は点 $(x,y)$ で全微分可能という. 
\end{definition}

\begin{proposition}{全微分可能性の十分条件}{prop:total-diff}
  $u,v$ が $D$ 上で $C^1$ 級ならば, $f$ は $D$ 上で全微分可能である. 
\end{proposition}

\begin{tproof}{命題の証明}{prf:total-diff}
  \begin{pstep}{1}
    $u$ が $C^1$ 級であるとき, ある連続な関数 $\varepsilon_u(\Delta x,\Delta y)$ が存在して
    \begin{mathpad}
      \[
        u(x+\Delta x,y+\Delta y)=u(x,y)+u_x\Delta x+u_y\Delta y+\varepsilon_u(\Delta x,\Delta y)\,r
      \]
    \end{mathpad}
    が成立し, $r\to 0$ で $\varepsilon_u\to 0$ となる. $v$ についても同様に
    \begin{mathpad}
      \[
        v(x+\Delta x,y+\Delta y)=v(x,y)+v_x\Delta x+v_y\Delta y+\varepsilon_v(\Delta x,\Delta y)\,r
      \]
    \end{mathpad}
    が成立する. 
  \end{pstep}
  \begin{pstep}{2}
    以上より
    \begin{mathpad}
      \[
        f(x+\Delta x,y+\Delta y)-f(x,y)
        =
        \begin{pmatrix}
          u_x & u_y\\
          v_x & v_y
        \end{pmatrix}
        \begin{pmatrix}
          \Delta x\\
          \Delta y
        \end{pmatrix}
        +
        \begin{pmatrix}
          \varepsilon_u\\
          \varepsilon_v
        \end{pmatrix}
        r
      \]
    \end{mathpad}
    となるので, 右辺の第2項が $o(r)$ であることが分かる. よって $f$ は全微分可能である. 
  \end{pstep}
  (\blacksquare) %chktex 12
\end{tproof}

\begin{figure}[H]
  \centering
  \includegraphics[width=0.8\linewidth]{img/fig4-1.pdf}%chktex 8
  \caption{全微分による一次近似のイメージ}\label{fig:fig4-1}
\end{figure}

\subsection{ヤコビ行列と微分の意味}
全微分可能なとき, 行列
\begin{mathpad}
  \[
    \mathrm{Jac}_f(x,y)\:=\:
    \begin{pmatrix}
      u_x & u_y\\
      v_x & v_y
    \end{pmatrix}
  \]
\end{mathpad}
を $f$ のヤコビ行列とよぶ. このとき
\begin{mathpad}
  \[
    f(x+\Delta x,y+\Delta y)-f(x,y)\approx \mathrm{Jac}_f(x,y)
    \begin{pmatrix}
      \Delta x\\
      \Delta y
    \end{pmatrix}
  \]
\end{mathpad}
となり, $f$ は微小なベクトルをほぼ線形変換として写す. 

\begin{proposition}{接ベクトルの対応}{prop:tangent-map}
  曲線 $\gamma(t)=(x(t),y(t))$ が $t_0$ で滑らかで, $f$ が点 $(x(t_0),y(t_0))$ で全微分可能ならば
  \begin{mathpad}
    \[
      \dv{}{t}f(\gamma(t))\Bigm|_{t=t_0}
      =
      \mathrm{Jac}_f(x(t_0),y(t_0))
      \begin{pmatrix}
        x'(t_0)\\
        y'(t_0)
      \end{pmatrix}
    \]
  \end{mathpad}
  が成り立つ. すなわち接ベクトルはヤコビ行列で写る. 
\end{proposition}

\begin{figure}[H]
  \centering
  \includegraphics[width=0.8\linewidth]{img/fig4-2.pdf}%chktex 8
  \caption{接ベクトルの対応}\label{fig:fig4-2}
\end{figure}

\subsection{複素微分可能性と微分}
複素関数の微分は, 実二変数の微分よりも厳しい条件を課す. 

\begin{definition}{複素微分可能性}{def:complex-differentiability}
  点 $z_0$ の近くで定義された複素関数 $f$ に対し
  \begin{mathpad}
    \[
      \lim_{h\to 0}\dfrac{f(z_0+h)-f(z_0)}{h}
    \]
  \end{mathpad}
  が存在するとき, $f$ は $z_0$ で複素微分可能であるという. この極限値を $f'(z_0)$ と書き, $f$ の微分係数とよぶ. 
\end{definition}

\begin{proposition}{コーシー・リーマンの必要条件}{prop:cr-necessary}
  $f(z)=u(x,y)+iv(x,y)$ が $z_0=x_0+iy_0$ で複素微分可能ならば, $u,v$ の偏導関数が存在して
  \begin{mathpad}
    \[
      u_x(x_0,y_0)=v_y(x_0,y_0),\quad u_y(x_0,y_0)=-v_x(x_0,y_0)
    \]
  \end{mathpad}
  が成り立つ. さらに
  \begin{mathpad}
    \[
      f'(z_0)=u_x(x_0,y_0)+iv_x(x_0,y_0)=v_y(x_0,y_0)-iu_y(x_0,y_0)
    \]
  \end{mathpad}
  が成立する. 
\end{proposition}

\begin{tproof}{命題の証明}{prf:cr-necessary}
  \begin{pstep}{1}
    $h$ を実数とすると
    \begin{mathpad}
      \[
        \dfrac{f(z_0+h)-f(z_0)}{h}
        =
        \dfrac{u(x_0+h,y_0)-u(x_0,y_0)}{h}
        +i\dfrac{v(x_0+h,y_0)-v(x_0,y_0)}{h}
      \]
    \end{mathpad}
    であるから, $h\to 0$ により
    \begin{mathpad}
      \[
        f'(z_0)=u_x(x_0,y_0)+iv_x(x_0,y_0)
      \]
    \end{mathpad}
    を得る. 
  \end{pstep}
  \begin{pstep}{2}
    $h=ik$ ($k\in\mathbb{R}$) とおけば
    \begin{mathpad}
      \[
        \dfrac{f(z_0+ik)-f(z_0)}{ik}
        =
        \dfrac{v(x_0,y_0+k)-v(x_0,y_0)}{k}
        -i\dfrac{u(x_0,y_0+k)-u(x_0,y_0)}{k}
      \]
    \end{mathpad}
    であり, $k\to 0$ から
    \begin{mathpad}
      \[
        f'(z_0)=v_y(x_0,y_0)-iu_y(x_0,y_0)
      \]
    \end{mathpad}
    を得る. 2つの表示を比較してコーシー・リーマン方程式が従う. 
  \end{pstep}
  (\blacksquare) %chktex 12
\end{tproof}

\begin{example}{$f(z)=z^2$ の微分}{ex:complex-derivative-square}
  $f(z)=z^2$ は全ての $z$ で複素微分可能であり
  \begin{mathpad}
    \[
      f'(z)=2z
    \]
  \end{mathpad}
  が成立する. 
\end{example}

\end{flushleft}
\end{document}
